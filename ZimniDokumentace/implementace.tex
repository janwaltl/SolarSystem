% !TeX spellcheck = cs_CZ

\chapter{Návrh programu}
\label{chap:implementace}
Tato kapitola se zabývá celkovým návrhem programu - celkovou strukturou a také popisuje jeho hlavní třídu - \texttt{Simulation}. V rámci které si představíme \textbf{moduly}, které se objevují v dalších kapitolách.
Na úvod se podívejme jak by mohl vypadat jednoduchý výsledný \textbf{main.cpp} 

\includecode{Source/main.cpp}{Hlavní program}

Veškeré zdrojové kódy programu jsou zabaleny do jmenného prostoru \texttt{solar}. \texttt{Simulation} je hlavní třída, která se stará o celkový průběh simulace. Dále jsou zde vidět tři \textbf{moduly}, které jsou předány simulaci. Jejich detailní popis přijde později.

\texttt{Simulation} poté zajišťuje jejich vzájemnou spolupráci. Metoda \texttt{Simulation::Start()} následně spustí samotnou simulaci podle předepsaných parametrů.
Pokud někde nastane chyba, dojde k vyvolání výjimky v podobě třídy \texttt{solar::Exception}, popřípadě jiných výjimek z rodiny \texttt{std::exception}.


% !TeX spellcheck = cs_CZ

\section{Třída \texttt{Simulation}}
Je třída, která spojuje jednotlivé moduly do funkčního celku a zajišťuje průběh celé simulace, proto stojí za to se na ni podrobněji podívat. Nejprve se podíváme na veřejné rozhraní, které dovoluje simulaci ovládat. Poté nahlédneme i dovnitř abychom zjistili jak to celé funguje.
\subsection{Veřejné rozhraní}
Přibližně takto vypadají veřejné metody třídy \texttt{Simulation} :
\includecode{Source/simPublic.cpp}{Veřejně rozhraní \texttt{Simulation}}
\paragraph{Řídící funkce}
jsou funkce, které řídí spuštěnou simulaci. Jsou volány z modulů, ne z hlavního programu. Většinou se jedná o jednoduché funkce na 1-2 řádky, proto zde nejsou jednotlivě popsány, ale z jejich názvů je zřejmé co dělají. Případný náhled do zdrojových kódů by to měl objasnit.
\paragraph{Konstruktor}
očekává 3 moduly, které se budou v simulaci používat, podrobněji se na ně podíváme za okamžik. Zde jim také předá odkaz na simulovaná data.
\paragraph{Metoda \texttt{Start()}}
provádí samotnou simulaci. V této metodě stráví program většinu času, popíšeme ji podrobně v sekci \ref{sec:startMetoda} - vysvětlíme její teoretický návrh a implementaci, což také objasní její parametry.
\subsection{Moduly}
\label{sub:moduly}
Nyní si vysvětlíme co to \textbf{moduly} jsou a k čemu slouží.
Zkusme se zaměřit co by měla každá simulace vlastně udělat:
\begin{enumerate}
	\item \textbf{Načíst data}, bez nich není co simulovat. Což se dá popsat dvěma slovy a implementovat stovky způsoby. Takže by stálo za to, aby simulace uměla všechny.
	\item \textbf{Simulovat data}. Z teorie už víme, že také neexistuje pouze jedna integrační metoda, takže bychom chtěli mít na výběr. Určitě totiž není moc šťastné řešení zvolit jednu metodu a doufat, že bude stačit na všechny simulace.
	\item \textbf{Uložit data}. Také se chceme našimi výsledky pochlubit, ale kdyby nám je simulace pracně získala a hned zahodila, tak to půjde těžko. Ovšem stálo by za to umět je ukládat v různých formátech nebo třeba rovnou odeslat někam po internetu.
	\item \textbf{Prohlížet data}. Představme si situaci: 3:38 ráno, naše značně vylepšená verze simulace právě doběhla. Přidali jsme do ní nově objevené asteroidy u kterých chceme spočítat trajektorie. Z hrůzou ale zjistíme, že Země není tam kde má být, dokonce tam není vůbec! Co se mohlo asi stát? To je dobrá otázka, proto by mohlo být dobré mít přístup k datům i během simulace a třeba si je někam průběžně ukládat.
\end{enumerate}
Teď už víme, co by měla každá simulace zvládnout, ale je vidět, že toho má umět celkem hodně a ještě různými způsoby. Nejlepší tedy bude, aby se simulace(ve formě třídy \texttt{Simulation}) opravdu starala jen o to, aby tyto kroky proběhly, ale to jak přesně proběhnou přenecháme někomu jinému -\textbf{ modulům}, které se vyskytují ve 3 druzích:
\begin{description}
	\item[parser] obstarává výrobu vstupních dat. Také na konci simulace může výsledná data uložit.
	\item[simMethod] provádí simulaci dat např. pomocí nějaké numerické metody.
	\item[viewer] má přístup k datům za běhu simulace a může je např. zobrazovat na obrazovku nebo ukládat stranou.
\end{description}
Práci jsme rozdělili, simulace by to celé měla tedy organizovat, což se děje právě pomocí metody \texttt{Start()}, tak si ji pojďme představit.
\subsection{Návrh metody \texttt{Start()}}
\label{sec:startMetoda}
Takto nějak by mohla vypadat na první pohled rozumná implementace:
\includecode[startPseudo]{Source/startPseudo.cpp}{Návrh metody \texttt{Start()}}
Tato implementace by byla plně funkční, ale možná ne úplně vhodná pro náš cíl. Rádi bychom totiž prováděli a hlavně zobrazovali simulaci v reálném čase. 

Zde ale není žádné časování, \texttt{while} smyčka bude probíhat jak nejrychleji může, bez jakékoliv kontroly. Což není špatné, pokud chceme nechat program běžet a jen se pak podívat na výsledky, popřípadě mezivýsledky uložené někde v souborech. K tomu přesně slouží metoda \texttt{Simulation::StartNotTimed()}, která má přibližně výše uvedenou implementaci. 
\paragraph{Časování}
- K dosažení našeho cíle bude potřeba nějakým způsobem svázat reálný čas s tím simulovaným. Upravíme tedy předchozí příklad \ref{lst:startPseudo} následovně:
\includecode[startTimed]{Source/startPseudoTimed.cpp}
{Metoda \texttt{Start()} s časováním}
Pokud bychom chtěli opravdu simulaci v reálném čase, tak se v každém průběhu smyčkou podíváme jak dlouho trvala předchozí smyčka a tolik času musíme odsimulovat. Naivní implementace by byla předat přímo tento čas \texttt{acc} metodě, která se stará o simulaci. Tím bychom dostali nedeterministický algoritmus
\footnote{Algoritmus, který nemusí nutně vracet stejný výsledek při opakovaném volání se stejnými vstupními hodnotami.}.
Neboť trvání poslední smyčky je ovlivněno např. aktuálním vytížením počítače, což určitě není předvídatelné. A bohužel při počítaní s desetinnými čísly dochází k zaokrouhlovacím chybám, takže výsledek nezávisí pouze na vstupních datech, ale i na postupu výpočtu.

Řešení \footnote{Zdroj: \url{http://gafferongames.com/game-physics/fix-your-timestep/} (\today)}
je naštěstí jednoduché, budeme odečítat pevnou hodnotu \texttt{deltaT}.
A to tolikrát, abychom odsimulovali potřebný čas uložený v \texttt{acc}. Je pravda,
že nakonci nemusí platit $ acc=0 $, ale bude platit $ acc\leq deltaT $. A vzhledem k tomu, že \texttt{acc} si zachovává hodnotu mezi průběhy smyčkou, tak se zbytek času neztrácí, ale použije se v dalším průchodu. Takto dostaneme deterministický algoritmus.

\label{par:spiral}
Navíc máme objasněn první parametr - \textbf{deltaT(dt)} - základní časový krok simulace, který odpovídá $ \Delta t $ z teorie. Čím menší, tím je simulace přesnější, ale dochází k více voláním simulační metody, což může být potencionálně náročné na CPU.
Pro nízké hodnoty by se klidně mohlo snadno stát, že simulace přestane stíhat, tzn. že simulovat čas \texttt{deltaT} bude reálně trvat déle než \texttt{deltaT}. Např.
simulace \texttt{10ms} bude konstantně trvat \texttt{15ms}, v dalším průběhu smyčkou se tedy bude simulace snažit simulovat uběhlých \texttt{15ms}, což ale může trvat \texttt{22,5ms}. V dalším průběhu se tedy musí odsimulovat \texttt{22,5ms}...takový případ velmi rychle celý program odrovná. Proto je vhodné proměnnou \texttt{acc} omezit nějakou konstantou. Poté sice začne docházet ke zpožďování simulace, ale kontrolovaným způsobem.
\paragraph{Změna rychlosti} - Nyní bude naše simulace fungovat zcela správně a v reálném čase. Takže máme hotovo? Skoro, naše simulace je sice plně funkční, ale jediné co umí je předpovídat přítomnost a ještě jen s omezenou přesností. To není moc užitečné. Oběh Země kolem Slunce bude skutečně trvat 1 rok a Neptun to zvládne za 165 let. Tolik času nejspíše nemáme, proto by stálo za to najít nějaký způsob jak simulaci urychlit. Existují dva způsoby jak to udělat:
\begin{enumerate}
	\item Volat simulační metodu častěji. Například pro každý krok \texttt{deltaT} ji můžeme zavolat \texttt{rawMult}krát. Toto zrychlení jde na úkor výpočetního výkonu nutného k udržení této rychlosti, viz. předchozí odstavec.
	\item Volat simulační metodu s jiným \texttt{deltaT}, konkrétně s jeho \texttt{DTMult} násobkem. Tohoto zrychlení dosáhneme na úkor přesnosti. Protože předáváme větší $ \Delta t $ do simulační metody, což vede k menší přesnosti.
\end{enumerate}
Parametry \texttt{rawMult} a \texttt{DTMult} přesně odpovídají argumentům implementované verze metody \texttt{Simulation::Start()}.

Poslední parametr, který nebyl ještě vysvětlen je \texttt{maxSimTime}. Vzhledem k tomu, že volání funkce \texttt{Start()} může trvat velmi dlouho, tak je dobré nastavit horní limit, který garantuje přerušení smyčky po překročení zadaného času. \texttt{maxSimTime=0} znamená \texttt{maxSimTime=$ \infty $} ,tzn. simulace se přeruší pouze zavoláním funkce \texttt{Simulation::StopSimulation()}, kterou ale mohou volat pouze moduly, neboť jiné objekty nejsou při simulaci volány.
Popřípadě se přeruší vyvoláním nějaké vyjímky.

\subsection{Finální implementace metody \texttt{Start()}}
Poučeni z předchozí části upravíme naši rozpracovanou implementaci \ref{lst:startTimed} na \ref{lst:startFinal}. Což je už velmi podobné skutečné metodě použité v programu. Která navíc umožňuje simulaci pozastavit, krokovat a také pomocí C++ knihovny \texttt{chrono} implementuje opravdové časování, které zde bylo uvedeno spíše koncepčně.
\includecode[startFinal]{Source/startFinal.cpp}
{Finální verze \texttt{Simulation::Start()}}



\section{Výjimky}
Existují tři druhy výjimek, které program může vyvolat.
\begin{enumerate}
	\item Nejčastěji to bude výjimka v podobě třídy \texttt{Exception}, která dědí 
	z \texttt{std::runtime\_error} a tedy i z \texttt{std::exception}. K této výjimce dojde zejména při chybě v otevírání/čtení souborů a inicializace knihoven, ale jedná se o hlavní třídu výjimek v tomto programu, takže je použita i v modulech pro hlášení chyb.
	\item Na výjimku v podobě třídy \texttt{GLError} můžeme narazit při chybě týkající se OpenGL - nepodařená inicializace, nedostatek paměti a podobně.
\item Výjimky dědící z \texttt{std::exception} - program využívá standardní C++ knihovny, které můžou potencionálně také vyvolat výjimky.
\end{enumerate}
Všechny tři druhy nabízí metodu \texttt{what()}, která vrátí krátký popis s chybou.