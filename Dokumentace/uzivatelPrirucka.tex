\chapter{Uživatelská příručka}
\label{chap:userGuide}
\section{Požadavky}
Vyžaduje verzi OpenGL 3 nebo vyšší, navíc je potřeba MVS redist, které je ve formě .dll přibaleno. Pokud by byly s program problémy, pak doporučuji stáhnout  a nainstalovat oficiální redist balíček z XXX.
\section{Základní ovladání}
Při spuštění .exe souboru se program otevře v výchozím grafickém režimu, kde dojde k nahrání Sluneční soustavy. Simulace se ovládá pomocí grafického rozhraní - hlavní je okno vlevo nahoře, které slouží k pozastavení simulace a úpravě její rychlosti. Další okna zobrazují informace o objektech a stavu simulace.\\
*Obrázek s vysvětlivkama*\\
\section{Pokročilé možnosti}
Program také nabízí pokročilejší ovládání pomocí příkazové řádky, s níž je možné načítat simulovaná data ze souboru,také nabízí možnost zaznamenat a následně přehrát uložené simulace.

*Vypsání obsáhlejší nápovědy*

K .exe souboru je přiložen vzorový formátovaný text a také jedna uložená simulace.
Přesný popis formátovaného vstupního souboru je uveden v sekci \ref{sec:strukturaDat} . Pro detailní vysvětlení parametrů simulace je k dispozici popis v sekci \ref{sec:startMetoda}.



\chapter{Kompilace programu}
\section{Git}
Tato dokumentace spolu se zdrojovými kódy programu je veřejně dostupná na:
\begin{center}
\texttt{https://bitbucket.org/Quimby/solar/src}
\end{center}
Popřípadě přímo stažitelná pomocí git:
\begin{center}
	\texttt{https://Quimby@bitbucket.org/Quimby/solar.git}
\end{center}
\section{Windows}
Visual studio projekt...knihovny už jsou, stačí zkompilovat
\section{Linux}
Potřeba zkompilovat knihovny dle návodu, překopírovat do libraries.
Udělat makefile?