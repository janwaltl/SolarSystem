\chapter{FormattedFileParser}
Tento \textbf{parser} načítá strukturovaná data z textového souboru. 

Parser očekává základní SI jednotky - metry, sekundy, kilogramy. Ale výstupní \texttt{simData\_t} obsahuje objekty se vzdáleností v AU($ 1,5.10^9 $ m), časem v rocích(pozemské - 365 dní) a hmotností v násobcích hmotnosti Slunce($ 1,988435.10^{30} $ kg přesně). Důvod je ten, že hodnoty v těchto jednotkách jsou více normalizované a dochází k menším zaokrouhlovacím chybám.

\section{Struktura dat}
\label{sec:strukturaDat}
Začneme hned příkladem, takto by mohl vypadat správný vstupní soubor s daty:
\begin{lstlisting}
{ name<Sun>
color<1.0 0.88 0.0 1.0>
position<0.0 0.0>
velocity<0.0 0.0>
mass<1.98843e30> }

{ name<Earth>
color<0.17 0.63 0.83 1.0>
position< 149.6e9 0.0>
velocity<0 29800.0>
mass<5.9736e24> }

{ name<Moon>
color<0.2 0.2 0.2 1.0>
position< 150.0e9 0.0>
velocity<0 30822.0>
mass<7.3476e22> }
\end{lstlisting}
\texttt{FormattedFileParser} by z tohoto souboru vytvořil data obsahující 3 objekty pojmenované - Sun, Earth, Moon s určitými vlastnostmi.

Každý objekt je popsán uvnitř páru složených závorek \textbf{\texttt{\{\}}}.
Všechny parametry objektu jsou dobrovolné, čili \lstinline|{}| je validní bezejmenný objekt, který se nachází v klidu na pozici \texttt{(0,0)} a má nulovou hmotnost. Pokud se ale parametr objeví, musí mít správný formát, který je obecně \texttt{název<hodnoty>}. Následuje přesný výčet a formát parametrů:
\paragraph{name$ < $jméno$ > $ } - jméno objektu. Jsou dovoleny pouze znaky ASCII, čili bez diakritiky. Pokud není uvedeno, pak je prázdné.
\paragraph{color$ < $R G B A$ > $} - barva objektu. Očekávají se 4 desetinná čísla v rozmezi od 0.0 do 1.0 oddělená mezerou představující barvu ve formátu RGBA. Pokud není uvedeno, pak je bíla.
\paragraph{velocity$ < $X Y$ > $} - počáteční rychlost objektu. Očekává dvě desetinná čísla oddělený mezerou reprezentující rychlost ve vodorovném a svislém směru. Nesmí zde být napsány fyzikální jednotky - tzn. \texttt{velocity$ < $10e2 m/s 8e3 m/s$ > $} \textbf{není} validní vstup. Ale implicitně by hodnoty měly být v m/s . Pokud není uvedeno, pak je (0,0).
\paragraph{position$ < $X Y$ > $} - počáteční pozice objektu. Identické s rychlostí, očekávají se hodnoty v metrech.
\paragraph{mass $ < $hmotnost$ > $ } - hmotnost objektu vyjádřená jedním desetinným číslem v kilogramech. Doporučení je zde také neuvádět jednotky, i když to bude fungovat, tak budou ignorovány. Pokud není uvedeno, pak je nulová hmotnost.
\paragraph{Dodatek k číslům} - jsou povoleny jak celá, tak desetinná čísla. Je dovolena pouze desetinná tečka. Číslo 104.25 můžeme zapsat například takto: $ 104.25 ;\ 1.0425e2$.
\textbf{Parser} ovšem \textbf{neumí} aritmetiku, takže následující \textbf{není} validní vstup: $ 10425/100 ;\ 104 + 0.25 ;\ 1.0425 * 10^2 $.
\paragraph{}
\texttt{FormattedFileParser} je relativně shovívavý k formátování, takže následující příklad je ekvivalentní definice objektu \texttt{Sun} z předchozího příkladu.
\begin{lstlisting}
{ Main star of our Solar system has a name <Sun>. 
  It's color is usually <1.0 0.88 0.0 1.0 which means yellow in RGBA format.
  The Sun accounts for 99% mass of our entire system,
  which makes it nearly 330 thousand times heavier than Earth
  or to put it in another way it weights < 1.98843e30 kilograms>. }
\end{lstlisting}

\section{Detaily implementace}
Takto vypadá deklarace konstruktoru: \\
\lstinline|FormattedFileParser(const std::string& inputFileName,const std::string& outputFileName="");|\\
Parser tedy očekává vstupní a případně výstupní jméno souboru,\textbf{ včetně} cesty a přípony k souboru. Samotné načítání a ukládání probíhá v metodách \texttt{Load()} respektive \texttt{Save()}.

Na vlastní implementaci není koncepčně nic extra zajímavého. \texttt{Load()} načte celý soubor do \texttt{std::string} ve kterém si pak vždy najde dvojici \{\}, ve které se pokusí najít názvy parametrů. Pokud nějaký najde, pak k němu ještě najde nejbližší dvojici $ <> $  ve které poté očekává správné hodnoty. Tento jednoduchý způsob zajišťuje výše ukázanou flexibilitu formátování. Všechno ostatní, co by se v dokumentu mohlo nacházet(například komentáře) prostě ignoruje. 

\texttt{Save()} vytvoří soubor, pokud byl nějaký zadán, do kterého předaná data uloží. Což se děje podobným způsobem - každý parametr "zabalí" do správného formátu a převede do jednotek SI.
Při nekorektním vstupu, nemožnosti otevřít vstupní soubor nebo vytvořit soubor výstupní vyvolají funkce výjimku \texttt{Exception}.


\chapter{SolarParser}
\section{Popis}
Tento jednoduchý \textbf{parser} nedělá nic jiného, než načte data o Sluneční soustavě, která má zabudována pevně ve své implementaci.
Momentálně se jedná o následující objekty:
\begin{enumerate}
	\item Hvězdy - Slunce
	\item Planety - Merkur, Venuše, Země, Mars, Jupiter, Saturn, Uran, Neptun
	\item Trpasličí planety - Pluto
	\item Měsíce - Měsíc, Phobos, Deimos, Io, Europa, Ganymed, Callisto
\end{enumerate}
Výstupní \texttt{simData\_t} obsahuje objekty se vzdáleností v AU($ 1,5.10^9 $ m), časem v rocích(pozemské - 365 dní) a hmotností v násobcích hmotnosti Slunce($ 1,988435.10^{30} $ kg přesně). Důvod je ten, že hodnoty v těchto jednotkách jsou více normalizované a dochází k menším zaokrouhlovacím chybám.

Také dokáže uložit výsledky simulace do formátovaného textového souboru stejného jako \texttt{FormattedFileParser}.