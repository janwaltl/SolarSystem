\chapter{Rozšířitelnost a co se dalo udělat lépe.}
\textbf{Poznámka autora: }\textit{Následující sekce popisuje rozšíření programu o novou funkci a to přehrávání uložených simulací. I když je tato myšlenka plně implementována a je funkční, tak celkem zneužívá design třídy \texttt{Simulation}, což nebylo nutné. Lepší verze je vložena jako sekce \ref{sec:extV2} \nameref{sec:extV2}, ale rozhodl jsem se zde ponechat i původní verzi. }
\section{Praktický příklad rozšíření programu V1.0}

Pokud jste už zkusili výsledný program spustit a samou radostí nad skvělou simulací vám něco uniklo a vy jste v panice začali hledat tlačítko na vrácení o krok zpět, tak jste zjistili, že tam žádné není. Bohužel program v aktuální verzi neprovádí žádné cachování výsledků, takže se nelze vrátit zpět. Což je určitě užitečné rozšíření, ale my zkusíme něco trochu jiného - \textbf{přehrávač simulací}.
Přehrávač by měl umět zaznamenat probíhající simulaci a poté ji přehrát jako video. Tedy včetně přeskakování na libovolné místo simulace, zastavení a zpětné přehrávání.
Jak by se takový přehrávač dal implementovat?
Co kdybychom vytvořili následující třídy:
\begin{enumerate}
	\item Třída \texttt{ViewAndRecord}, která se chová jako \texttt{viewer}, ale navíc simulaci zaznamenává do souboru. 
	\item Trojice tříd \texttt{ReplayerParser} \texttt{ReplayerMethod} a \texttt{ReplayerViewer}, které by se starali o přehrávání simulace.
\end{enumerate}
\ref{lst:extIdea} ukazuje výše zmíněnou myšlenku převedenou do C++ a  \ref{lst:extIdeaUsage} pak jak by se pak dala použít v hlavním programu.
\includecode[extIdea]{Source/extIdea.cpp}{Návrh přehrávače simulací} 
\includecode[extIdeaUsage]{Source/extIdeaUsage.cpp}{Příklad použití přehrávače}
Díky návrhu celého programu se zaznamenávání simulace docílí velmi jednoduše, protože jediné co musíme změnit je, že "zabalíme" zvolený \textbf{viewer} do třídy \texttt{ViewerAndRecord} a poté ho předáme jako obyčejný \textbf{viewer} simulaci. Zabalení znamená, že ho předáme jako parametr pro šablonu \texttt{ViewerAndRecord}. Při běhu simulace bude pak volán \texttt{ViewerAndRecord}, který ale také zavolá předaný viewer a navíc bude na pozadí ukládat probíhající simulaci.

Přehrávání pak docílíme tím, že simulaci budeme podstrkovat data, která si přečteme ze souboru místo toho abychom je simulovaly. Tento podvod bude zajišťovat právě třída \texttt{ReplayerMethod}. Technicky potřebujeme ještě \textbf{parser} - \texttt{ReplayerParser}, ale ten jediné co udělá je, že přečte první data ze stejného souboru. Pro zobrazení použijeme třídu \texttt{ReplayerViewer}, který využívá pracně vytvořený \texttt{IMGuiViewer}. Pomocí knihovny ImGui můžeme pak dokreslit potřebné ovládání simulace.

Bohužel narazíme na problém. \texttt{Simulation} neumí měnit libovolně čas. \texttt{deltaT} je pevně dané, takže nemůžeme simulaci přehrávat pozpátku. Což se dá vyřešit tím, že čas simulace budeme do jisté míry ignorovat a \texttt{ReplayerViewer} si přímo řekne \texttt{ReplayerMethod}, že by měl nyní vracet data pozpátku.
\section{Praktický příklad rozšíření programu V2.0}
\label{sec:extV2}