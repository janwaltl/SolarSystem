% !TeX spellcheck = cs_CZ

\chapter{Úvod}
\label{chap:uvod}

\section{Specifikace}
Program simuluje Sluneční soustavu za využití numerické integrace a Newtonova gravitačního zákona. Fyzikálně se jedná řešení problému n-těles - tzn. každé těleso gravitačně působí na všechna ostatní. Tento problém je velmi těžko řešitelný analytickým metodami pro větší n. Výpočetní síla počítačů spolu s metodami numerické integrace tak nabízí alternativní řešení tohoto problému.

\paragraph{}
Vstupem programu jsou strukturovaná data uložená v textovém souboru, která definují fyzikální veličiny simulované soustavy.Tedy polohy, rychlosti a hmotnosti simulovaných objektů. 
\paragraph{}
Výstup je 2D grafická reprezentace simulované soustavy v reálném čase. Uživatelské rozhraní dovoluje měnit rychlost a přesnost simulace. Dále zobrazuje užitečné informace, jako jsou aktuální pozice, rychlosti pro každý simulovaný objekt vzhledem k jiným objektům.
\paragraph{}
Při vývoji programu byl kladen co největší důraz na pozdější rozšířitelnost. Výsledný program tedy poskytuje několik simulačních metod a možností vstupů a výstupů. 

\section{Teorie}
\subsection{Analytický popis}
Newtonův Gravitační zákon (dále NGZ) popisuje vzájemné silové působení $ {F}_g $ dvou hmotných bodů
\footnote{Myšlené těleso, kde jeho veškerá hmotnost je soustředěna do jednoho místa - \textbf{hmotného bodu}. }
, kde výsledná síla je přitažlivá.

\begin{equation}
	{F}_g= \kappa \dfrac{m_1 m_2}{(\boldsymbol{x_1 - x_2})^2} 
\end{equation}
$ m_1,m_2 $ jsou hmotnosti obou bodů a $ \boldsymbol{x_1,x_2} $ jejich polohy.

Dále budeme pokládat simulované objekty za hmotné body, což je vzhledem k rozměrům hvězd, planet, měsíců a jejich vzdálenostem rozumná aproximace.

Nyní nám NGZ spolu s principem superpozice
\footnote{Princip superpozice říká, že výsledné silové účinky na těleso jsou dány součtem všech sil, které na něj působí.}
 a Newtonovým Zákonem síly \eqref{eq:sila} dává pro $ n $ těles následující \eqref{eq:soustava} soustavu  $ n $ obyčejných diferenciálních rovnic. 
 Kde neznámé $ \boldsymbol {x}_i, \boldsymbol{\ddot x}_i $ jsou vektory polohy, resp. zrychlení simulovaných těles. Mínus je zde kvůli přitažlivosti výsledné síly.
\begin{equation}
F= m  \boldsymbol {\ddot x}
\label{eq:sila}
\end{equation}
\begin{align}\label{eq:soustava}
\boldsymbol {\ddot x}_i = -\kappa \sum_{j=1,j \neq i}^{n}\dfrac{m_i\left( \boldsymbol{x_i}(t) - \boldsymbol{x_j}(t)\right)}
{\left( \boldsymbol{x_i}(t) - \boldsymbol{x_j}(t)\right) ^3} . 
\quad \text{pro } i=1 \dots n
\end{align}

Analytické řešení této soustavy rovnic by nám dalo možnost zjistit polohu, rychlost a zrychlení libovolného simulovaného objektu v libovolném čase na základě počátečních podmínek.
\paragraph{}
Bohužel vyřešit tuto soustavu je pro $ n=>3 $ velmi těžké.

\subsection{Numerické řešení}
Pokud se soustava nedá vyřešit analyticky, můžeme se alespoň pokusit získat aproximativní řešení. Numerická integrace využívá toho, že nemusí být těžké spočítat derivace v libovolném čase. Navíc to dnešní počítače dokáží udělat velmi rychle. Pokud tedy dokážeme zjistit derivaci v každém bodě, tak bychom mohli původní funkci zrekonstruovat pomocí těchto derivací. Např. můžeme výslednou funkci aproximovat úsečkami, kde jejich směrnice je derivace hledané funkce. Toto přesně dělá nejjednodušší integrační metoda - Eulerova explicitní metoda\eqref{eq:euler}.
Mějme jednoduchou soustavu rovnic \eqref{eq:ex} . Je vidět, že řešením je funkce$ y=e^t $, což se snadno ověří zpětnou derivací. 
\begin{align} \label{eq:ex}
\dot	y = e^t, y(0)=1
\end{align}
Zkusme tuto rovnici vyřešit numericky Eulerovou metodou \eqref{eq:euler}. 
Výsledné hodnoty jsou uvedeny v tabulce \ref{tab:numer}, kde bylo použito postupně $ \Delta t = 1;\> 0.1;\> 0.01$ a také analytické řešení.
\begin{align} \label{eq:euler}
	y(t+\Delta t) = y(t) + \Delta t . y'(t)
\end{align}
\begin{table}[h!]
	\centering
	\label{tab:numer}
	\begin{tabular}{c||c|c}
	$ \boldsymbol t $ & $\boldsymbol{ e^x} $  &\textbf{	 Euler}\\
		
		\hline
		1 & 1 & 1\\
		1.1 & 1.254 & 1.255
	\end{tabular}
\caption{Caption for the table.}
\end{table}
\paragraph{}
Z tabulky vidíme, že pro dostatečně malé $ \Delta t$ dostáváme dostatečně přesné hodnoty.
Eulerova metoda se dá použít i k řešení naší soustavy \eqref{eq:soustava}, což je podrobněji popsáno u její implementace v sekci \ref{sec:implEuler}. Také další integrační metody - semi-implicitní Euler a RK4 jsou popsány u svých implementací \ref{sec:implRK4}.










