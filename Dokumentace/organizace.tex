% !TeX spellcheck = cs_CZ

\chapter*{Organizace dokumentu}
Tento text je organizován do následujících částí:
\begin{description}
	\item[Úvod] - První kapitola, zde je uvedena specifikace zápočtového programu a vysvětlena teoretická část.
	\item[Programátorká část] - Kapitoly č.X až Y, které popisují jak design celého programu, tak jednotlivých částí. Zaměřují se na použité algoritmy a jejich implementaci včetně zdrojových kódů C++. Na konci jsou poté zmíněny možnosti dalšího rozšíření.
	\item[Uživatelská příručka] - Poslední dvě kapitoly popisují jak program správně zkompilovat a také jak s ním pracovat z neprogramátorského pohledu.
	
\end{description}
Programátorská část vyžaduje určitou znalost C++ a OpenGL. Důležité koncepty a algoritmy jsou podrobně vysvětleny a vhodně doplněny zdrojovými kódy, které jsou psány následujícím formátem:
\begin{lstlisting}[title=Text vystihující příklad (Název souboru)]
#include <iostream>

int main()
{
	std::cout<<"Hello World!\n";
	return 0;
}
\end{lstlisting}
Všechny další uvedené zdrojové kódy se nachází ve složce \texttt{Source/}, která
by měla být připojena k této dokumentaci. Z praktických důvodů \textbf{nemusí} být tyto soubory zkompilovatelné, popřípadě mohou být psány z části v pseudo-kódu. \textbf{Nejedná} se o zdrojové kódy samotného programu. Primární účel je pouze \textbf{popisný}, kde snaha je ilustrovat koncepty a algoritmy, které se v programu v nějaké formě skutečně vyskytují.

Zdrojové kódy samotného programu jsou také součástí dokumentace a obsahují komentáře, které mohou stát za přečtení.

