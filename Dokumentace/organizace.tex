% !TeX spellcheck = cs_CZ

\chapter*{Organizace dokumentu}
Tento text je organizován do následujících částí:
\begin{description}
	\item[Úvod] - První kapitola, zde je uvedena specifikace zápočtového programu a také teoretický popis problému.
	\item[Programátorká část] - Kapitoly č.2 až 8, které popisují jak návrh celého programu, tak jednotlivých částí. Zaměřují se na použité algoritmy a jejich implementaci včetně ukázek zdrojových kódů v C++. Na konci jsou poté zmíněny možnosti dalšího rozšíření.
	\item[Kapitola \ref{chap:compMethods} ] nabízí srovnání použitých algoritmů pro simulaci.
	\item[Kapitoly \ref{chap:userGuide} a \ref{chap:kompilace}] popisují jak s programem pracovat z neprogramátorského pohledu a také jak ho zkompilovat.
\end{description}
Programátorská část vyžaduje určitou znalost C++ a OpenGL. Důležité koncepty a algoritmy jsou podrobně vysvětleny a vhodně doplněny zdrojovými kódy, které jsou psány následujícím formátem:
\begin{lstlisting}[title=Název vystihující příklad]
#include <iostream>

int main()
{
	std::cout<<"Hello World!\n";
	return 0;
}
\end{lstlisting}
Z praktických důvodů nemusí být tyto ukázky zkompilovatelné, popřípadě mohou být psány z části v pseudo-kódu, nejedná se tedy přímo o zdrojové kódy samotného programu. Primární účel je popisný, kde snaha je ilustrovat koncepty a algoritmy, které se v programu v nějaké formě skutečně vyskytují.

\chapter*{Zdroje a Reference}
Můj hlavní zdroj informací byl předmět NOFY056 Programování pro fyziky 2015/2016 vedený RNDr. Ladislavem Hanykem, Ph.D. a cvičení s Mgr. Tomášem Ledvinkou, Ph.D., kde se vyučují mimo jiné právě numerické metody řešení ODR. 

Zvlášť užitečné byly poznámky z \url{http://geo.mff.cuni.cz/~hanyk/NOFY056/index.htm} k výše zmíněné přednášce. Hlavně pro pochopení RK4 a také jsou zde přehledně napsány vícekrokové metody prediktor-korektor. Každopádně ale i anglická verze Wikipedie má celkem dobře sepsané všechny zde použité numerické metody.
