% !TeX spellcheck = cs_CZ

\chapter*{Organizace dokumentu}
Tento text je organizován do následujících kapitol:
\begin{description}
	\item[\nameref{chap:uvod}] - Zadání zápočtového programu a teoretická část
	\item[\nameref{chap:implementace}] - Hlavní část dokumentace, které popisuje jak design celého programu, tak jednotlivých částí. Zaměřuje se na použité algoritmy a jejich implementaci včetně zdrojových kódů C++.
	Také zmiňuje možnosti rozšíření programu.
	\item[\nameref{chap:userGuide}] - Část popisující jak program spustit a jak s ním pracovat z neprogramátorského pohledu.
	
\end{description}
Text není nutné číst od začátku do konce, pro první spuštění by mělo stačit poslední kapitolu. Naopak, ale pro pochopení implementace RK4 metody je dobré vědět něco o numerické integraci a o co se vlastně program vůbec snaží. Což popisuje teoritická část v úvodní kapitole. Programátorská část vyžaduje znalost C++ a OpenGL, ale je zde snaha důležitější koncepty vysvětlit  i bez těchto znalostí. V tomto textu se nachází  zdrojové kódy s příklady, které jsou psány následujícím formátem:
\begin{lstlisting}
\caption{Text vystihující příklad (Název souboru)}
#include <iostream>

int main()
{
	std::cout<<"Hello World!\n";
	return 0;
}
\end{lstlisting}
Všechny další uvedené zdrojové kódy se nachází ve složce \texttt{Source/}, která
by měla být připojena k této dokumentaci. Z praktických důvodů \textbf{nemusí} být tyto soubory zkompilovatelné, popřípadě mohou být z části v pseudo-kódu. Také se \textbf{nejedná} o zdrojové kódy samotného programu. Primární účel je \textbf{popisný}. Ovšem často bude uvedený kód přímo nebo částečně odpovídat kódu někde uvnitř programu.
Zdrojové kódy samotného programu se přímo v tomto dokumentu nenachází, avšak jsou také součástí dokumentace a obsahují komentáře, které mohou stát za přečtení.

