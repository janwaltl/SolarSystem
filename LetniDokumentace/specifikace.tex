% !TeX spellcheck = cs_CZ

\chapter{Úvod}
\label{chap:uvod}

\setcounter{page}{1}
\section{Specifikace}
Program simuluje Sluneční soustavu za využití Newtonova gravitačního zákona a numerických metod. Fyzikálně se jedná řešení problému n-těles - tzn. každé těleso gravitačně působí na všechna ostatní. Tento problém je velmi těžko řešitelný analytickým metodami pro větší n. Výpočetní síla počítačů a numerické metody tak nabízí alternativní řešení tohoto problému.

Vstupem programu jsou strukturovaná data uložená v textovém souboru, která definují fyzikální vlastnosti simulované soustavy, což případně dovoluje snadné změny v zadání. 

Výstup je 3D grafická reprezentace simulované soustavy v reálném čase. Uživatelské rozhraní dovoluje plynule měnit rychlost a přesnost simulace. Dále zobrazuje užitečné informace, jako jsou aktuální pozice a rychlosti pro každý simulovaný objekt vzhledem k jiným objektům. Celou simulaci je také možné nahrát a poté kdykoliv přehrát pomocí zabudovaného přehrávače.

Při vývoji programu byl kladen co největší důraz na pozdější rozšířitelnost. Výsledný program tedy poskytuje několik simulačních metod a lze jej lehce rozšířit o další metody, popřípadě vstupy/výstupy.

\section{Rekapitulace}
\label{chap:rekapitulace}
Tato kapitola nabízí rychlý přehled návrhu celého programu. Detailnější popis je v kapitole 2 ze zimního semestru.


